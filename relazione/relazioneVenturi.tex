\documentclass[10pt, a4paper]{article}
%-----------------------
%- 	PACKAGES & SETTINGS
%-----------------------
\usepackage[a4paper,top=1.5cm,bottom=2cm,left=4cm,right=4cm]{geometry}
\usepackage[utf8]{inputenc}
\usepackage[italian]{babel}
\usepackage{xcolor}
\usepackage{hyperref}
\hypersetup{
    colorlinks=true,
    filecolor=magenta,      
    urlcolor=darkgray,
    linkcolor=black
}
\urlstyle{same}
\usepackage{amsmath}
\usepackage{graphicx}
\graphicspath{ {images/} }
 
%-----------------------
%- 	TITLE
%-----------------------
\title{\textbf{Relazione Progetto OCaml PR 2}}
\author{\textbf{Venturi} Ludovico\\Docente: \href{http://pages.di.unipi.it/levi/}{Francesca Levi}}
\date{UNIPI, Gennaio 2020}


%-----------------------
%- 	DOCUMENT
%-----------------------
\begin{document}
%- 	INTRO
\pagenumbering{roman} 
\maketitle
\tableofcontents
\vfill
\begin{figure}[h]
	\centering
	\includegraphics[scale=0.3]{ocaml_logo}
	\label{fig:0}
\end{figure}

\clearpage

%- 	START DOC
\pagenumbering{arabic} 
\section{Scelte progettuali}

\section{Utilizzo}
Per utilizzare aprire l'interprete top-level di OCaml da terminale digitando \texttt{ocaml}. Importare l'interprete del nostro linguaggio:
\begin{center}
\texttt{\# use "venturi.ml";;}
\end{center}
e da qui valutare le espressioni, entrando nel vivo del \textbf{REPL}(ReadEvalPrintLoop).
\subsection{Esempi}

\end{document}